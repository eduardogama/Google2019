\section*{Related Work}
\label{sec:releated-work}

Zhang~\textit{et al.}~\cite{zhangINFOCOM17} focuses on the client-side, performing the average bitrate level by the bitrate adaptation algorithm and the influence of chunk size variation for improve the QoE, while Shen~\textit{et al.}~\cite{shenIWQoS19} works with a set of cache proxy services to analyze the cache miss occurrences. This work implements a reactive approach where cache proxies download the chunks of multimedia content when requested. Rosário~\textit{et al.}~\cite{rosarioSENSORS2018} describes a multi-tier environment that provides a live migration service from the cloud to the different tiers. The experimental scenarios delivery a video stream between different tiers with QoE support.
Poliakov~\textit{et al.}~\cite{poliakovPHD2018} deploy a DASH video streaming with multiple  sources. The DASH-client player can download the chunks, at the same time, through different connections on the cloud.
% Cache-aware load balancing of data center applications
Archer \textit{et al.}~\cite{archerGoogleJournal2019} proposes an algorithm to deal with the cache replicas for flash bandwidth video provisioning, which is a critical bottleneck.

The aforementioned approaches could decrease the traffic load and improve QoE, but more issues arise in Smart City scenarios: user mobility, collaborative cache schemes over multi-edge, the amount of users during flash crowds, and interactive streaming requirements are not fully considered. In this project we aim to design a video delivery system that considers such issues to improve quality of experience for a range of video streaming needs, including low latency requirements.
