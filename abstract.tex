\begin{center}
	{\large {\bf Abstract}}
\end{center}
\noindent
Video Streaming based on Dynamic Adaptive Streaming over HTTP~(DASH) has been widely adopted by video providers such as Google, Netflix, Akamai HD and others, where the client-side video player can dynamically choose the bitrate level according to the perceived available bandwidth. 
At the same time, video streaming services represent the majority of the internet traffic, and according to Cisco forecasts\footnote{Cisco Visual Networking Index: Global Mobile Data Traffic Forecast Update. Link:~\url{http://shorturl.at/hjAZ1}. Accessed: July 29, 2019.}, in 2021 70\% of all internet traffic will be dominated by video streaming. This includes current video services as well as innovative services such as cloud gaming and future consoles (e.g. Google Stadia), whereas for mobile devices this estimate represents 78\% of all mobile data traffic. To accommodate video traffic, a good cloud-level architecture partially solves some issues related to the live stream and Video on Demand~(VoD) services. However, a centralized cloud service introduces some issues such as higher latency and core network congestion. Therefore, to improve video services, it is of paramount importance to properly distribute video streams according to their requirements: a cloud gaming infrastructure is an interactive service that needs reduced delays (a few milliseconds), while a non-interactive VoD delivery can tolerate higher delays without impairing quality of experience. A proper management and orchestration of video delivery over the Internet is core to the smooth co-existence of heterogeneous video services. This project proposes the use of edge/cloud hierarchy to design a cooperative DASH video streaming in Smart Cities, deploying cache service to offer improved Quality of Experience~(QoE) for end-users. 